We next compare group lasso, GrOWL-I and GrOWL-II by analyzing synthetic data
generated from a deep neural network model trained to generate distributed
representations of a word's sound (phonology) and meaning (semantics) from its
spelling (orthography; Figure~\ref{fig.network} top left). The network structure
is motivated by the influential ``triangle'' model of the human reading system
\citep{PlautETAL96}. Specifically, phonological outputs receive contributions
from two separate pathways: a {\em direct} route mediated by a single hidden
layer, and an ``indirect'' route composed of three hidden layers, which must
first compute mappings from orthography to semantics, then project onward to
contribute to the phonological outputs. This architecture is interesting because
different kinds of similarity structure emerge through learning in different
network components. The central idea is that orthographic and phonological
similarities are highly systematic: items that are similar in spelling are
likely (though not guaranteed) to be similar in pronunciation. These
regularities are easily learned within the direct pathway mapping from
orthography to phonology, allowing the system to generate appropriate
pronunciations for previously unseen word-forms. In contrast, orthographic and
semantic similarity structures are unsystematic: similarity of word spelling
does not necessarily predict similarity of meaning and vice versa. In learning
to map from orthography to semantics and on to phonology, the indirect path thus
comes to encode quite different similarity relations amongst the words than does
the direct path \citep{PlautETAL96,HarmSeidenberg04}.

To capture these properties we generated model ``orthographic'' representations
as patterns sampled from 6 overlapping clusters of binary input features,
roughly corresponding to different orthographic neighborhoods. For every word a
``phonological'' pattern was generated by flipping each orthographic feature
with probability 0.1. Thus phonological patterns were distorted variants of
orthographic patterns, creating high systematicity between these. We also
created a ``semantic'' pattern for each word from a set of binary features also
organized into clusters. Across items, these vectors expressed a hierarchical
similarity structure with two broad superordinate clusters each composed of
three tighter clusters. Importantly, the similarity structure expressed by the
semantic vectors was independent of the structure expressed in the
orthographic/phonological patterns. The left bottom panel in
Figure~\ref{fig.network} shows the cosine distances encoded amongst the 30
``words'' in each layer of one trained model. Layers in the direct path each
encode roughly the same distances amongst items, while the semantic layer
encodes a quite different set of distances that is weakly reflected in two of
the three hidden layers in the indirect path.  Thus the different components of
this simple word-reading network contribute differentially to the encoding of
semantic versus ortho-phonological similarity structure.

We trained 5 models with different initial weights, corresponding to 5 model
subjects, and presented each with 30 orthographic inputs. Each input generated a
vector of activations over the 100 model units. To ensure high redundancy
amongst units this vector was concatenated 5 times and perturbed with
independent noise, yielding 500 measurements per model subjects. These were
treated as analogs of the estimated BOLD response at each of 500 model voxels in
a brain imaging study. We then applied group lasso and GrOWL to find the voxel
subsets that encode either the semantic or phonological distances (derived from
the target values for the semantic and phonological output layers of the
network). We fit models by searching a grid of parameters ($\lambda$,
$\lambda_1$), including $\lambda_1=0$ as the special case of GrOWL that is group
lasso. For each grid point we counted a voxel as ``selected'' if it received a
non-zero weight, and assessed how accurately the model selected  the voxels
encoding phonological structure (all those along the direct pathway) or semantic
structure (the semantic layer hidden layers 2 and 3 in the indirect path) by
computing hit rates and false alarm rates. All three models showed low and
equivalent cross-validation error; however GrOWL-II achieved this error rate
while selecting considerably more voxels. The ROC plots in Figure~\ref{fig.roc}
further show that GrOWL-II did not select additional voxels at random: it
outperformed group LASSO considerably in discriminating signal-carrying from
non-signal carrying voxels. The right panel of Figure~\ref{fig.network} shows
the frequency with which each model unit is selected for the best-performing
solution of each method and structure type. The strong sparsity enforced by
group lasso is clearly apparent: target units are selected less consistently
than with GrOWL, which consistently discovers more of the signal.

Finally, we considered the ability of GrOWL to reveal the network structure
encoding each kind of similarity, treating the weights in the matrix $\bW$ as
direct estimates of the joint participation of pairs of units in expressing the
target similarity. The rightmost plots of Figure~\ref{fig.network} show the
estimated connectivity, thresholded to show the 25\% of the non-zero weights
with the largest magnitudes. The detected edges clearly express the network
representational substructure: units in the direct pathway are shown as highly
interconnected with one another and weakly or disconnected from those in the
indirect pathway, and vice versa. Thus the search for different kinds of
similarity reveals different functional subnetworks in the model.

\begin{figure*}[!t]
\centering
\includegraphics[width=0.99\textwidth]{figures/Network_results1.png}
\caption{Left panel: Network architecture (top) and the similarity structure
  expressed in each layer (bottom). Red background shows the direct pathway and
  blue the indirect pathway from orthography to phonology. Layers in the two
  pathways encode different similarity structures. The target similarity
  matrices for the analysis express either the semantic structure (top layer) or
  the phonological structure (bottom right layer). Arrows indicate feed-forward
  connectivity. Right panel: Units selected by group LASSO (right) and GrOWL
  (middle) when decoding semantic (top) or phonological (bottom) structure.
  Colors show the proportion of times across subjects and unit concatenations
  that the unit received a non-zero weight, with red indicating 1 and gray 0.
  The rightmost plots show the largest weights in the associated matrix W for
  each GrOWL model, which pick out two subnetworks in the model.}
\label{fig.network}
\end{figure*}

\begin{figure*}[!h]
\centering
\subfloat[]{\includegraphics[width=0.4\textwidth]{figures/ROC_sem.pdf}
\label{fig_first_case_roc}}
\hfill
\subfloat[]{\includegraphics[width=0.39\textwidth]{figures/ROC_phon.pdf}
\label{fig_second_case_roc}}
\caption{ROC curves generated by sweeping through $\lambda$ values (for $\lambda
  = 0$, all units are selected and as $\lambda$ is increased fewer units are
  given non-zero weight).  Each curve represents a fixed value of $\lambda_1$,
  where the curve $\lambda_1 = 0$ corresponds to group lasso. ROC curves are
  averaged across participants for each method, considering both similarity
  structures, Semantics (left panel) and Phonology (right panel).}
\label{fig.roc}
\end{figure*}
